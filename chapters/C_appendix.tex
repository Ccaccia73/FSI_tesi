\chapter{preCICE API}
\label{app:pc-api}

\section{preCICE API calls}
\label{sec:api-code}

Each participating solver needs to be linked to the \acrshort{precice} library and call methods from its \acrfull{api}. The preCICE adapter groups together the calls to the API. While preCICE is written in C++, it provides an API also for C, Fortran and Python. An excerpt from the C++ API is shown in Listing~\ref{precice-api-code} as drawn from the preCICE documentation.

A coupled simulation is first configured and initialized phase, then multiple coupling advancements occur up to a  finalization phase.

First a \texttt{SolverInterface} object needs to be created: its parameters are the rank of the process and the size of the communicator if the execution is parallel.

The \texttt{configure()} method defines the preCICE configuration file, reads it and sets up the communication inside the solver.
The following methods that follow in Listing~\ref{precice-api-code} are called ``steering methods''. The \texttt{initialize()} method sets up the data structures and communication channels to other participants.At the first communication the participants exchange meshes. It returns the maximum time step size that the solver is allowed to
execute next.

\texttt{initializeData()} optionally transfers any initial non-zero coupling data values among participants. The \texttt{advance()} method take care of equation coupling, data
mapping and communication.

\texttt{finalize()} destroys the data structures and closes the communication channels.

Each solver defines its interface mesh. Each mesh and each node on the mesh are assigned an integer ID. \texttt{getMeshID()} gets the ID of the mesh over which the coupling is performed. \texttt{setMeshVertex()} creates a vertex on the specified mesh position. Additionally, topological information, such as edges or triangles can be passed to preCICE with additional methods.

Data values are assigned to meshes by means of methods with names of the kind \texttt{write*Data()}, which fill the buffers with data from the solver's mesh or
with methods like \texttt{read*Data()}, which read data from the buffers into the solver's mesh.

Some other methods allow to access useful information for the coupling, in particular if the simulation is still running, if there are new data do be read or written or if the current coupling time step has converged or not.

The solver asks if it has to write or read a checkpoint and it informs preCICE that it fulfilled these tasks \cite{uekermann2016partitioned}.

\newpage


\lstset{language=C++}
\begin{lstlisting}[caption=preCICE API methods,label=precice-api-code]
class SolverInterface
{
public:
	SolverInterface(
		const std::string& solverName,
		int solverProcessIndex,
		int solverProcessSize);

	void configure(const std::string& configurationFileName);
	
	double initialize();
	void initializeData();
	double advance(double computedTimestepLength);
	void finalize();
	
	int getMeshID(const std::string& meshName);
	int setMeshVertex(int meshID, const double* position);
	void setMeshVertices(int meshID, int size, double* positions, int* ids);

	void writeScalarData(int dataID, int valueIndex, double value);
	void writeVectorData(int dataID, int valueIndex, const double* value);
	void writeBlockScalarData(
		int dataID,
		int size,
		int* valueIndices,
		double* values);

	bool isCouplingOngoing();
	bool isReadDataAvailable();
	bool isWriteDataRequired(double computedTimestepLength);
	bool isTimestepComplete();

	bool isActionRequired(const std::string& action);
	void fulfilledAction(const std::string& action);
// ...
};
\end{lstlisting}

\newpage

\section{preCICE adapter structure}
\label{sec:adapter-code}

An example of an adapted solver is shown in Listing \ref{adapter-struct}, as published in the presentation article of preCICE \cite{bungartz2016precice} and also in the preCICE website\footnote{\href{https://github.com/precice/precice/wiki/Adapter-Example}{precice/wiki/Adapter-Example}}

\lstset{language=C++}
\begin{lstlisting}[caption=preCICE adapter structure,label=adapter-struct]
turnOnSolver(); //e.g. setup and partition mesh 

precice::SolverInterface precice("FluidSolver","precice-config.xml",rank,size); // constructor

int dim = precice.getDimension();
int meshID = precice.getMeshID("FluidMesh");
int vertexSize; // number of vertices at wet surface 

// determine vertexSize
double* coords = new double[vertexSize*dim]; // coords of vertices at wet surface 

// determine coordinates
int* vertexIDs = new int[vertexSize];
precice.setMeshVertices(meshID, vertexSize, coords, vertexIDs); 
delete[] coords;

int displID = precice.getDataID("Displacements", meshID); 
int forceID = precice.getDataID("Forces", meshID); 
double* forces = new double[vertexSize*dim];
double* displacements = new double[vertexSize*dim];

double dt; // solver timestep size
double precice_dt; // maximum precice timestep size

precice_dt = precice.initialize();
while (precice.isCouplingOngoing()){
	
	if(precice.isActionRequired()){
		saveOldState(); // save checkpoint
		precice.markActionFulfilled();
	}

	precice.readBlockVectorData(displID, vertexSize, vertexIDs, displacments);
	setDisplacements(displacements);
	dt = beginTimeStep(); // e.g. compute adaptive dt 
	dt = min(precice_dt, dt);
	computeTimeStep(dt);
	computeForces(forces);
	precice.writeBlockVectorData(forceID, vertexSize, vertexIDs, forces);
	precice_dt = precice.advance(dt);
	if(precice.isActionRequired()){ // timestep not converged
		reloadOldState(); // set variables back to checkpoint
		precice.markActionFulfilled();
	}
	else{ // timestep converged
		endTimeStep(); // e.g. update variables, increment time
	}
}
precice.finalize(); // frees data structures and closes communication channels
delete[] vertexIDs, forces, displacements;
turnOffSolver();
\end{lstlisting}
