\chapter{MBDyn Adapter and its integration}
\label{cha:adapter}

Objective

structure of the code (diagrams?)

- mbdynAdapter: connection to precice

- mbdynConnector: connection to MBDyn

input: json config file

- simulation parameters

output: VTK and resultants






%In order to couple SU2 with preCICE, a C++ adapter class named Precice1 is developed in the scope of
%this work. A header file precice.hpp and a source file precice.cpp are the practical outcome. The Precice
%class encapsulates all coupling related activities and separates them from the original SU2 source code.
%It makes use of the high-level API provided by preCICE. Since the adapter is integrated into the source
%code of SU2, it is completely compiled with it (for a description on how to install SU2 with preCICE,
%see Appendix B). This way, coupling is achieved with minimally invasive code changes in SU2 and an
%adaption of the original code is, thus, possible with only small effort, basically reduced to copy-paste
%tasks. The adapter allows for usage of both explicit and implicit coupling strategies implemented in
%preCICE and fully conforms with intra- and interfield parallelism. Moreover, usage of the adapter is
%assimilated to the regular configuration process of SU2, thus, it is embedded smoothly into the software
%suite. All options concerning the usage of preCICE (e.g. switching it on or off, specifying name and
%location of the preCICE configuration file, etc.) are set via the SU2 configuration file. Consequently, no
%recompilation of SU2 is necessary when the user decides to use/not to use preCICE. In addition, a single
%executable, SU2_CFD, is enough to account for single-physics simulations (without preCICE) as well as
%for FSI computations via preCICE. Figure 5.1 shows a schematic representation of the code coupling
%approach.
%Concerning notation of code shown in this chapter (and in the corresponding referenced sections of the
%appendix), it is important to state that SU2 uses several "containers" for storing information (technically,
%they are multiple pointers). E.g. a "config_container” is an instance of CConfig or a "geometry_container"
%refers to CGeometry. Furthermore, all shown code excerpts are reduced to the necessary information.
%Therefore, not all arguments of functions are stated but only the relevant ones. Also, ellipses (...) are
%used to denote further lines of code, which are not shown for the sake of simplicity.
%This chapter is organized in the following top-down way: Starting from the most user-respective changes
%in SU2, in Section 5.1, the newly added, coupling-related options included in the SU2 configuration
%file are presented and their usage is explained. In addition, necessary code changes are stated. The
%chapter continues with a more technical, detailed description on how the coupling is embedded in SU2,
%as in Section 5.2 adaptions to the main routine of SU2_CFD are described. Mostly, these modifications
%include calling several functions, which are incorporated in the adapter class Precice. However, the tasks
%hidden in these functions are not described until finally, the adapter itself is extensively explained with
%emphasis on both physical and computational details at the end of this chapter in Section 5.3. Referring
%back to Figure 5.1, Sections 5.1 and 5.2 correspond to "code changes" in SU2, while Section 5.3 relates
%to the "Coupling Adapter".
%5.1 Changes Concerning SU2 Configuration
%In order to fully control the usage of preCICE for FSI simulations within SU2, new options in the
%configuration file of SU2 are available (for a recapitulation of this file, see Section 4.2.2). They are listed
%in the following with their default values:
%PRECICE_USAGE = NO, YES (5.1a)
%PRECICE_CONFIG_FILENAME = precice.xml (5.1b)
%PRECICE_VERBOSITYLEVEL_HIGH = NO, YES (5.1c)
%PRECICE_WETSURFACE_MARKER_NAME = wetSurface (5.1d)
%Most obvious, PRECICE_USAGE is a flag used for determining whether a simulation should be run
%with or without preCICE. Modifying the remaining three options is only reasonable if it is set to YES.
%PRECICE_CONFIG_FILENAME specifies the name of the configuration file of preCICE. Also, its path
%relative to the location of the SU2 configuration file must be specified. In order to allow users to have more
%insight into the sequence of activities within the coupling adapter, the level of verbosity of the adapter can
%be chosen. If PRECICE_VERBOSITYLEVEL_HIGH is set to YES, several checkpointing information
%of the adapter is output to the console. Yet, too much console output can slow down simulations. Since
%this information is typically not relevant when running an FSI simulation productively, the verbosity level
%is chosen to be low by default. This feature of the coupling adapter is mainly included for tracing back
%runtime errors. Eventually, as explained in Section 4.2.2, physical boundaries are treated as markers in
%SU2. Each boundary has a unique identifying name, which is specified in the SU2 mesh file. The boundary
%marker name corresponding to the FSI interface of the fluid mesh must be passed to the coupling adapter,
%which is done via the option PRECICE_WETSURFACE_MARKER_NAME. A description on how to
%adapt SU2 in order to use the new configuration options is given in Listings A.1, A.2 and A.3, Appendix
%A to a detailed level.
%The dynamic mesh capabilities of SU2 must be enabled, in order to use the implemented ALE formulation
%of the flow solver. This is done via:
%GRID_MOVEMENT = YES (5.2)
%Still, a specific kind of grid movement needs to be chosen. Intrinsically available are e.g. specifications
%for rigid motions or rotations of the mesh. Here, a new option is available, which is mandatory if SU2 is
%used for FSI simulations via preCICE:
%GRID_MOVEMENT_KIND = PRECICE_MOVEMENT (5.3)
%A manual on how to add this new grid movement option to the configuration procedure of SU2 is given
%in Listing A.4, Appendix A.
%Yet, the implementation of PRECICE_MOVEMENT is missing. It defines the steps of which the mesh
%movement procedure consists. After displacements of the nodes at the wet surface are transferred to
%SU2 via preCICE, the mesh needs to be deformed and smoothed (for a reminder, see Sections 2.2.3 and
%3.3). Despite the sole mesh deformation, also grid velocities must be calculated in order to be able to
%use the ALE method of SU2. Finally, for cases in which the SU2 multigrid capabilities are enabled,
%displacements and velocities of the mesh nodes need to be mapped to all grid levels. Intrinsic SU2
%mesh movement procedures cannot be reused as either they do not include the three necessary steps
%stated above (mesh deformation, grid velocity computation, forwarding information to multigrid levels)
%or they involve further computations, which are not necessary for FSI simulations and therefore, represent
%unnecessary computation overhead. The code defining the steps of PRECICE_MOVEMENT is given in
%Listing A.5, Appendix A.
%
%5.2 Adaption of SU2 Main Routine
%After modifying files related to the configuration procedure of SU2 in the previous section, it remains to
%adapt the SU2_CFD module in SU2_CFD.cpp, which relates to the solver run procedure itself. The goal
%is to add as little code as possible in the main solver routine of SU2 such that the coupling adapter can
%be used. Detailed, corresponding code excerpts are stated in Appendix A.
%One core criterion for integrating the adapter into SU2 is that the solver executable should be able
%to run both single- and multiphysics simulations without recompilation. Only a single adapter-related
%variable needs to be initialized in the main routine of SU2 regardless of whether preCICE is used for a
%simulation or not. The variable (called precice_usage) is a boolean flag corresponding to the newly added
%PRECICE_USAGE option of the SU2 configuration file. This flag is the basis for conditionally triggering
%all coupling activities. If it is set to false, no more coupling variables are initialized in SU2_CFD and
%the single-physics solver runs according to the regular scheme previously shown in Algorithm 2, Section
%4.2.2. The (only three) additional variables needed for coupling include an instance of the adapter class
%Precice and two time-stepping variables (namely max_precice_dt and dt).
%preCICE needs to be able to shut down SU2, in case the FSI simulation should be ended. Therefore, an
%adaption of the main solver while-loop is necessary. In case a simulation runs without preCICE, the usual
%condition of the while-loop is used, which checks that the number of solver iterations is smaller than a
%specified maximum. However, if a coupled FSI simulation is executed, the adapter additionally evaluates
%if preCICE signalizes a solver shut down.
%Assuming that the solver loop is executed, a checkpointing procedure for implicit coupling strategies of
%preCICE starts. For strongly coupled algorithms, preCICE tries to find the fixed-point of the coupling
%equation system (as explained in Section 4.1.1). If the solution of a subiteration does not satisfy the pre-
%CICE convergence criteria, resetting the fluid solver to the start of the time step is necessary. Therefore,
%at the beginning of each solver iteration in SU2, the current solver state is saved such that reloading it
%becomes possible. However, this is only done in the first subiteration of a new time step.
%As mentioned in Section 4.1, preCICE might need to enforce time step sizes for the single-physics solvers.
%To allow for the same in SU2, the minimally allowed time step size for an iteration needs to be determined
%before the solution procedure starts. Here, the variables dt and max_precice_dt come into play. While
%the former stores the current time increment of SU2, the latter is the maximum prescribed by preCICE.
%After SU2 is done with executing a single solver iteration, preCICE is informed that a new flow solution
%is available. Therefore, the coupling tool might advance in time. This triggers preCICE to manage the
%data exchange between SU2 and other coupled solvers, as well as to execute the coupling algorithm. In
%case of an implicit procedure, convergence acceleration techniques are also activated by this step.
%In case a strongly coupled algorithm is chosen, preCICE needs to evaluate (by checking its convergence
%criteria) whether the old solver state of SU2 needs to be reloaded, staying at the same physical time
%instance, or the simulation can proceed with the next time step. In the former case, SU2 internal solver
%variables are reset to the last time instance. It is not reasonable to allow SU2 to write output files if
%preCICE signalizes that the current time step has not sufficiently converged.
%Finally, SU2_CFD handles the clean shut down procedure at the end of an FSI simulation. Communication
%channels are closed via the adapter and coupling-related memory is deallocated.
%In Appendix A the extended solver procedure of SU2_CFD is depicted in Algorithm 4 by analogy with
%the original solver sequence shown in Algorithm 2.
%
%5.3 Coupling Adapter
%As shown in the previous section, most code changes in SU2_CFD consist of conditional clauses checking
%whether the adapter is used, followed by function calls on the adapter object. In this section I explain
%the adapter functions and how they relate to preCICE. No code excerpts are included in this section, as
%the files precice.hpp and precice.cpp will soon be included in the open-source preCICE repository. Thus,
%the interested reader is referred to the source code.
%The adapter makes use of the high-level API provided by preCICE. Its main component is an interface
%with predefined functions that need to be integrated in the adapter. The corresponding class (in preCICE)
%is called SolverInterface. Simply calling functions of this class within SU2 is not sufficient for coupling.
%Rather, the adapter also takes care of
% force calculation at the FSI interface,
% managing intrafield parallelization of SU2 in the coupling process,
% converting data from SU2 to preCICE specific representation and vice versa,
% setting up and triggering mesh deformation,
% as well as reading and writing iteration checkpoints.
%All these functionalities are smoothly hidden within the adapter class. Directly integrating these tasks in
%SU2 would imply highly invasive code changes in its main routines. Yet, the adapter class also contains
%some functions, which I refer to as wrappers. They consist of not much more but function calls on
%SolverInterface. The advantage of this technique is obvious: There is no need to instantiate an object of
%class SolverInterface directly in SU2. Rather, only the adapter instantiates such an object and therefore
%hides it from the main solver routines. Table 5.1 gives an overview of functions2 of the adapter class
%Precice and whether they are wrappers or not.
%function name wrapper function?
%configure() yes
%initialize() no
%advance() no
%isCouplingOngoing() yes
%isActionRequired() yes
%getCowic() no
%getCoric() no
%saveOldState() no
%reloadOldState() no
%finalize() yes
%Table 5.1: Functions of the adapter class Precice and their characterization.
%Some aspects of the adapter functions are already mentioned in Section 5.2. However, detailed explanations
%of what these functions do and their connection to preCICE is eventually given in the following.
%
%Startup of a Coupled Simulation
%The whole coupling process starts with the instantiation of the adapter within the main solver routine of
%SU2. Upon creation of the adapter object, several information is passed to it, including MPI rank and
%size, as well as all geometry (CGeometry), solver (CSolver), configuration (CConfig) and grid movement
%(CVolumetricMovement) related data. Next to initializing data structures needed for coupling, the most
%important step is the instantiation of a SolverInterface object within the adapter, which represents the
%adapter’s connection to preCICE (compare Figure 5.1).
%The adapter object and its connection to preCICE are established, yet it still remains to configure
%preCICE from its configuration file. This happens when configure() is called on the adapter with the
%name and location of the configuration file as input argument. Since this function is a wrapper, internally
%the adapter calls the same function on SolverInterface and forwards name and location of the .xml file.
%Consequently, preCICE parses the configuration file and creates necessary data structures for coupling.
%Subsequently, communication between SU2 and its coupling partner, as well as preCICE internal meshing
%at the wet surface needs to be initiated, which happens upon calling initialize() on the adapter. It checks
%each node at the FSI interface and stores its coordinates in a data array, which is then forwarded to
%preCICE via the SolverInterface object. It is important to keep in mind that intrafield parallelism is
%possible in SU2. The corresponding domain decomposition procedure (for a recapitulation, see Section
%4.2.3) can lead to situations, in which a process does not work on the wet surface at all. This is taken into
%account in the adapter as follows: As mentioned in Section 5.1, the boundary marker name of the wet
%surface must be given to SU2 during configuration. This name is now used to determine whether a process
%includes FSI interface nodes or not. Consequently, the respective process is marked by a boolean flag,
%processWorkingOnWetSurface. If it evaluates to false, the before mentioned coordinate transfer procedure
%is skipped. After receiving the node coordinate information, preCICE is prepared to create an interface
%mesh from it. Finally, initialize() is executed on the SolverInterface object, which triggers setting up
%communication between the coupling partners, creating the wet surface mesh and computing a possible
%restriction on the first time step of SU2.
%
%Coupling Step
%The actual coupling activities start with the main solver loop of SU2. As explained in Section 5.2, a solver
%iteration only occurs if preCICE signalizes that the coupling should not be stopped yet. Therefore, in the
%wrapper function isCouplingOngoing() a function of the same name is executed on the SolverInterface
%object and its boolean return value serves as the demanded signal. Subsequently, at the beginning of an
%iteration and in case an implicit coupling algorithm is chosen, preCICE needs to inform SU2 whether
%the current iteration corresponds to the first one of a new time step. isActionRequired() is called on
%the adapter, which is again just a wrapper for the same function call on SolverInterface. The input
%argument, however, specifies whether the action refers to writing or reading an iteration checkpoint. For
%this purpose, the adapter includes two constant string member variables. One corresponds to writing
%and one to reading a checkpoint. They can be accessed by the respective getter-functions getCowic()
%and getCoric(). In this case, the former is chosen since the adapter might have to write an iteration
%checkpoint.
%If so, saveOldState() is called on the adapter. The goal of saving the current ("old") solver state is to
%store all information, which is necessary to be able to rerun exact the same iteration, implying that an
%iteration of SU2 with the same computational outcome is expected, if the input is not changed. This is
%important for implicit coupling algorithms of preCICE if convergence is not met and a time step needs
%to be restarted. Then, preCICE varies the input of SU2 in terms of the transferred displacements, which
%might yield a better result (in terms of forces), meaning that the residuals of the coupled fixed-point
%system are reduced. The quantities, which need to be stored for this checkpointing strategy, include
%variables associated with the nodal coordinates of the fluid mesh, the grid movement and the solution of
%previous time steps.
%After a solver iteration of SU2 advance() is called on the adapter object. This function is the adapter’s
%most extensive one as it includes computing forces at the wet surface and transferring them to preCICE.
%Moreover, it triggers the coupling algorithm in preCICE, as well as receiving and setting the nodal
%displacements at the FSI interface computed by the structural solver. There is no predefined function
%available in SU2, which computes forces at certain mesh nodes. Therefore, I implemented this computation
%in the advance() function. First of all, the adapter again checks whether the respective MPI process works
%on the wet surface or not via processWorkingOnWetSurface. Computing and forwarding forces is only
%necessary for the nodes in immediate contact with the FSI interface. If a process includes wet surface
%nodes, the adapter determines the kind of flow regime of SU2 (compressible or incompressible, viscous or
%inviscid flow). As mentioned in Section 4.2.1, SU2 is currently not able to run incompressible simulations
%with ALE support. However, I already include the force computation for incompressible flows in the
%adapter, in case this capability will be added in future releases. In addition, the adapter computes a
%factor for redimensionalizing forces (as SU2 features non-dimensionalized simulations as well). If the
%current simulation is dimensional, this factor evaluates to 1. In order to explain the force calculation, I
%assume the general, three-dimensional case of a simulation governed by the NSE. The overall force at a
%node acting on the FSI interface is then given by a viscous term arising from the viscous stress tensor
%
%and an inviscid term determined by the (dynamic) pressure. The computation is done as follows:
%fi = 􀀀(ptotal 􀀀 pstatic)niA + ijnjA 8i = 1; 2; 3; with (5.4a)
%ij = (
%@vi
%@xj
%+
%@vj
%@xi
%) 􀀀
%2
%3
%
%@vk
%@xk
%ij 8i; j = 1; 2; 3: (5.4b)
%f denotes the force vector, p pressure (total and static, respectively) and  the viscous stress tensor. A
%and n refer to area and unit outward normal vector of the dual mesh element associated with the node, for
%which the force is calculated. Note that the pressure term needs to be negated as by definition pressures
%point inward the fluid control volume, but the pressure force exerted on the solid (i.e. outward the fluid
%control volume) is required. The explanation of the viscous stress tensor in Equation 5.4b is identical to
%Equation 2.6. Again, the adapter needs to manage intrafield parallel execution of SU2. As extensively
%explained in Section 4.2.3, ghost nodes are introduced in SU2 in order to build halo-cells after domain
%decomposition. If decomposition occurs at the wet surface, interface nodes are replicated. Allowing each
%of the replicates and the original nodes to write forces to preCICE would yield unphysical computation
%results, as those nodes share the same mapping to the solid mesh and therefore, forces would accumulate
%at solid nodes. The adapter makes use of the colors assigned to the FSI interface nodes in SU2. This is
%technically done by comparing the MPI rank (= color of the process) with the colors of the wet surface
%nodes. If they do not match, the process works on a duplicate and thus, is not allowed to write forces
%at such a node. The corresponding data array storing all FSI forces is eventually pushed to preCICE by
%calling writeBlockVectorData() on the SolverInterface instance.
%The next step is executing advance() on the SolverInterface object, which uses the length of the current
%solver iteration of SU2 as input and returns the prescribed maximum for the next iteration. Internally,
%preCICE now executes convergence acceleration techniques, if an implicit procedure is chosen and exchanges
%coupling data with the partner solvers.
%In the following, the adapter needs to read the FSI interface nodal displacements calculated by the coupled
%structural solver. Thus, readBlockVectorData() is called on the SolverInterface object. The so obtained
%displacements are set as coordinate variations relative to the nodal positions of the last time step in SU2.
%While writing forces must be restricted to the nodes originally belonging to a process, reading and setting
%displacements needs to be done also for the replicates.
%Although it would be possible to trigger the mesh deformation procedure right away, I decided against
%this strategy as the current time step might have to be restarted and mesh deformation would be unnecessary
%computational overhead in such a case. Thus, it is triggered at the beginning of the next solver
%iteration as shown in Algorithm 3.
%Now the counterpart of writing an iteration checkpoint comes into play (only for implicit algorithms).
%preCICE checks whether the fixed-point equation system converges sufficiently or not. In the latter case,
%upon calling the wrapper function isActionRequired() with input argument getCoric() on the adapter,
%the coupling tool signalizes that reloadOldState() needs to be executed. Consequently, the solver state
%prior to the current iteration is retrieved by resetting the respective variables in SU2.
%
%Clean Exit
%The main solver loop of SU2 is usually exited when isCouplingOngoing() in the while-loop condition
%evaluates to false, which means that preCICE tries to finish the FSI simulation. The last step of the
%coupling is initiated when the wrapper function finalize() is executed on the adapter object. This causes
%all communication channels related to the coupled simulation to be closed and used memory to be
%deallocated.

