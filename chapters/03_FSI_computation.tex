\chapter{Computational aspects of Fluid-Structure Interaction problems}
\label{cha:computation}

Subsequent to the basic physics of FSI problems drawn up in the last chapter, this section goes into detail
on computational treatment, which also allows for FSI techniques to be categorized. Two fundamentally
different classes of procedures for solving FSI simulations have been established, namely monolithic and
partitioned approaches. They are discussed in Section 3.1. Since this thesis focuses on the latter kind,
a characterization of weak and strong coupling in partitioned approaches is described in Section 3.2.
Subsequently, in Section 3.3, a different way of categorizing FSI techniques is presented, which is based
on conforming or non-conforming mesh treatment of respective solver strategies. The chapter is closed
by Section 3.4, having a look at the added mass effect (AME) - a stability issue arising from problems
with strong interaction in partitioned FSI simulations.

\section{Monolithic and Partitioned Approach}

First of all, I want to point out that the terms monolithic and partitioned are interpreted differently
throughout FSI literature. However, in this thesis, they are only used in the hereafter explained sense.
Again, let the name fluid-structure interaction serve as a motivation for this section: On the one hand,
it implies a single (but coupled) physical problem while on the other hand, its clear multiphysical characteristic
is emphasized. This is also reflected in the monolithic and partitioned approaches, respectively.
However, monolithic and partitioned approaches should not be interpreted as solely oppositional. There
exist solver strategies for which the boundaries between the two approaches blur.
Monolithically treating both fluid and solid domain implies that they are solved simultaneously. This
means that one multiphysics solver deals with a single system of equations describing fluid, solid and their
coupling. Figure 3.1 depicts this strategy schematically. Such monolithic solvers are designed specifically
for the sole purpose of solving FSI problems. Therefore, a high level of specialization can be realized.
Simultaneously treating both flow and structural equations often results in good numerical stability of
the calculations. Furthermore, monolithic approaches solve the system of equations exactly, meaning
there are no errors (other than those which are inherent to numerical techniques) introduced by this form
of numerical treatment. However, development of such solvers from scratch requires a lot of coding work
and is often cumbersome ([8], [38], [20], [18]).
In contrast, partitioned approaches make use of existing single-physics solvers. The FSI problem is split
into a fluid and solid problem, which are both treated by their respective solvers separately, while a third
software module, the coupling component, incorporates the interaction aspects. It communicates forces or stresses (dynamic data) calculated by the fluid solver at the wet surface to the solid component and
exchanges displacements or velocities (kinematic data) computed by the solid solver at the interface to
the fluid component in return ([18], [38]). A schematic sketch of this situation is shown in Figure 3.2.
More detailed and practical explanations about the coupling component are given in Section 4.1. For
now it is sufficient to know that the coupling component exchanges kinematic and dynamic data between
the single-physics solvers in order to preserve the coupled nature of the overall problem. In the end,
fluid and structural solutions together yield the FSI solution. By analogy with the monolithic approach,
it is graphically depicted in Figure 3.3. A big advantage of this approach is that existing solvers for
the fluid and solid problem can be reused, ranging from commercial to academic and open-source codes.
Especially in the commercial sector, these are often highly elaborate solvers with decades of experience in
their particular single-physics fields. They typically support very sophisticated solution techniques. Also,
those solvers are usually well-validated and compared to monolithic procedures the programming efforts
are lower for partitioned approaches, as only the coupling of the existing solvers has to be implemented
rather than the solvers themselves. Nevertheless, these advanced solvers can only be of good use for FSI
simulations if the coupling component is sufficiently precise ([8], [38], [20]).

\section{Coupling Strategies}

Partitioned strategies for solving FSI problems can be subclassed into weakly and strongly coupled approaches.
They are also referred to as explicit and implicit methods in this thesis. Note that this
nomenclature is in no way fully consistent throughout FSI literature. However, all usage of these terms
in this thesis is limited to the sense of the explanations given in this section.
The distinction between explicit and implicit techniques is based upon the question, how often solutions
for the fluid and solid subproblem are computed within one time step and also, how frequently the
relevant kinematic and dynamic quantities are exchanged. For weakly coupled strategies solving is done
a certain, fixed number of times (often only once) per time step and data may not even be communicated
after each discrete time instance. In general, this is not sufficient to regain the monolithic ("exact")
solution of the FSI problem as the coupling conditions are not enforced within each time step. Thus,
no balance between fluid and structural domain with respect to energy, forces and displacements at the
interface can be guaranteed ([8], [18], [38], [2]). However, this coupling strategy can still yield good
results if the interaction between fluid and solid is rather weak (further explanations about the strength
of the interaction follow in Section 3.4). E.g. in aeroelastic simulations, where small displacements of
the structure appear within single time steps, the flow field is influenced by the structural displacements
only to a little extent ([13], [8], [2]).
In contrast, strongly coupled strategies make use of subiterations possibly resulting in multiple computations
of the separate solvers and exchanges of the interface coupling quantities (as a reminder, see Figure 3.2) per physical time step. However, acceleration techniques are necessary to converge the underlying
coupling equation system. The coupling conditions at the wet surface are enforced in each time step up to
a convergence criterion. If the criterion is not met sufficiently, another subiteration within the same time
instance is calculated. Therefore, the solution can approximate the monolithic solution to an arbitrary
accuracy as the convergence criterion can be chosen as strict as needed. Such a method is in general
applicable to both FSI problems, which can be solved by weakly coupled approaches and those, for which
explicit procedures fail due to dominant interaction. However, strongly coupled algorithms are usually
used in the latter case - when weak coupling reaches its limits - since the implicit approach requires more
computational effort ([18], [38], [2]).
Figure 3.4 sums up the categorization of different coupling techniques mentioned in this and the previous
section. They are ordered with respect to stability and programming effort.

\section{Interface Mesh}

In this section, FSI methods are classified by means of two different mesh treatment procedures: conforming
or non-conforming techniques. The basic question is, whether fluid and solid mesh need to align
with each other at the FSI interface or not. Unless stated otherwise, the explanations of this section are
taken from [20]. Note that some aspects of conforming mesh methods are already included in the previous
sections without explicitly mentioning so, in order to develop a better understanding of partitioned FSI
simulations.
Conforming mesh methods usually consist of three major subtasks, namely computation in the fluid and
solid domain, as well as interface and mesh movement. They require both fluid and structural meshes
to conform to the wet surface, because the coupling conditions are applied via the interface as physical
boundary conditions for the respective domains. This does not necessarily imply node-to-node matching
of fluid and structure meshes at the interface. This must hold for all time instances, which means that
both fluid and structural grids need to be moved in case deformations of the solid appear. This is a
simple task for the solid mesh, since it is usually expressed in a Lagrangian fashion anyway. However, as
a typical Eulerian fluid mesh would not follow the interface motion, the necessity of the ALE method as
discussed in Section 2.2.3 becomes apparent. Also, mesh smoothing techniques need to be introduced in order to prevent quality losses of the fluid mesh in terms of distorted elements. These irregularities lead
to accuracy loss in simulations. In Figure 3.5, a conforming mesh is shown at two different points in time.
At the first instance (Figure 3.5a) the solid is undeformed and therefore, also the fluid mesh remains in
its initial configuration. In contrast, at the second point in time (Figure 3.5b) the solid is deformed and
the fluid mesh conforms to the displaced wet surface. Consequently, also mesh smoothing is applied.
There is a wide variety of such mesh updating procedures. Some common techniques compute the mesh
movement by considering mesh edges as springs ((torsional) spring analogy), solving the Laplace equation
or solving a pseudo-structural system of equations (see e.g. [18], [43], [20] and their respective references
for further explanations of these techniques). Conforming mesh strategies are widely, but not exclusively
used in partitioned FSI approaches. Furthermore, they typically also utilize the ALE method ([20]).
In contrast, in non-conforming mesh strategies all interface conditions are directly imposed as constraints
on the flow and structural governing equations. Therefore, it is possible to use non-conforming meshes
for fluid and solid domains as they remain geometrically independent from each other. Thus, also mesh
smoothing techniques are obsolete [43]. Figure 3.6 depicts such a situation. By analogy with Figure 3.5,
again the initial configuration (Figure 3.6a) and an instance when the solid is deformed (Figure 3.6b) are
shown. It is clearly visible that the fluid mesh does not conform to the wet surface as all nodes stay at
the same position regardless of the solid deformation.
This approach is mostly used in immersed methods. The considerations in this section are limited to them,
as they are also very common for FSI simulations. Coupling is imposed via additional force-equivalent
terms appearing in the model equations of the fluid, enforcing the kinematic and dynamic conditions.
These FSI forces are computed from the structural model, which is dealt with separately together with
tracking the position of the interface. The forces represent the effects of a boundary or body being
immersed in the fluid domain (leading to immersed boundary and immersed domain methods). A purely
Eulerian mesh can be applied to the whole computational domain for solving the fluid equations, since
the force-equivalent terms are dynamically added in a spatially specific manner to those locations, which
are currently occupied by the structure. After solving the fluid equations, forces exerted on the solid at
the wet surface are computed and used as input for the structural solver, which still employs a Lagrangian
mesh. Subsequently, the deformation of the solid material is calculated and the displacement of the FSI
interface is fed back to the fluid model in form of updated force-equivalent terms ([31], [20], [43]).

\section{Stability: Added Mass Effect}

To conclude this chapter about computational aspects of FSI simulations, the AME is briefly described.
Explanations of this effect can be found in a great variety throughout FSI literature, typically explicated
for specific solver strategies or flow regimes (see e.g. [5], [42], [15], [2]). Therefore, in the scope of
this thesis only a short phenomenological introduction to the concept of added mass and numerical
problems arising from it is given. However, this suffices to focus on both weakly and strongly coupled partitioned approaches, which are practically relevant in this thesis. The AME is inherent to partitioned
FSI approaches as the single-physics fields are not continuously coupled but interaction only occurs at a
finite number of discrete time instances, when coupling quantities are exchanged.
As already mentioned in Section 2.3.3, there can be no gaps between structure and fluid. Also their
respective particles cannot occupy the same spatial locations simultaneously. Thus, if the solid is moved,
also fluid particles move. Changing the state of motion of the structural component consequently requires
taking into account inertial effects not only of the solid itself, but also of the surrounding fluid, which
artificially rests for the span of a single structure solver time step. In more descriptive words: Moving the
solid also implies moving fluid particles close to the solid. Therefore, the structure behaves more inert
due to artificially added mass ([42], [2]). Since inertia is dependent on mass and therefore density, the
% AME is also. More precisely, it is dependent on the ratio (MA) of structural (S) und fluid density (F )
([42], [5]):
%MA = S F : (3.1)
This ratio is often used to describe how strong the interaction between solid and fluid is. For cases, in
%which the solid density is much higher than the fluid density (MA  1), this effect does not dominantly
influence the FSI problem (weak interaction). But as fluid and structure densities approach each other
%(MA  1) or the fluid becomes even denser than the solid (MA < 1), its consideration is crucial (strong
interaction) and imposes stability limits on partitioned solution techniques ([5], [42], [2]). Note that the
AME is not only governed by the density ratio of Equation 3.1 but also by geometric properties of the
problem, the stiffness of the solid ([5]) and the speed of sound in the flow domain ([42]). Nonetheless,
for the sake of simplicity and intuitiveness, explanations in this thesis are mostly limited to the density
ratio.
In general, the AME is of bigger concern for incompressible flows than for compressible regimes. From
a physical point of view, deformations of the structural domain can be interpreted as perturbations for
the flow field. In compressible flows the speed of sound (speed at which perturbations propagate through
the flow) is finitely large. Thus, the influence of a geometrical change of the fluid domain caused by
deformations of the solid is locally limited during a certain period of time. In contrast, in incompressible
flows the speed of sound is infinitely large, hence all perturbations propagate through the flow without
time delay. Therefore, regardless of how much time has passed since a perturbation, the whole flow field
is directly affected ([42], [5]).
In the following it is assumed that a weakly coupled algorithm allows computation of the fluid and solid
solution only once per time step. In addition, coupling data is also exchanged once per time instance. In
contrast, a strongly coupled solver does the same at least twice per time increment (for a reminder see
Section 3.2 and compare Figure 3.4). As can be shown, in the compressible case a more dominant AME
can be compensated for by reducing the time step size of strongly coupled, partitioned solution algorithms.
This however, does not hold for the incompressible regime, where even in the limit of vanishing time step
size strongly coupled, partitioned algorithms might fail ([42]). These observations are consistent with the
above mentioned physical explanation.
First of all, considering compressible flows, indeed, the lack of repeated subiterations in weakly coupled
partitioned techniques leads to a strict limit for the density ratio (of Equation 3.1) due to the fact that
the interface conditions are not enforced and energy balance at the wet surface is generally not given.
If that limit is exceeded the algorithm fails due to instability ([2]). Likewise, in such a case a strongly
coupled partitioned algorithm converges slowly, resulting in possibly many necessary subiterations, which
is computationally costly. Yet it does not become unstable, given that the time step size is chosen sufficiently
small. Reducing the time step size to an arbitrarily small extent cannot stabilize a weakly coupled
approach if the stability criterion on the density ratio is not met ([15]). Conversely, the convergence
rate of strongly coupled algorithms increases by the same factor, by which the time step size decreases,
meaning that in the limit of vanishing time step size the monolithic solution is obtained ([5], [42]).
In the incompressible case however, a strict stability limit exists for both weakly and strongly coupled
algorithms. It is independent of the size of time increments1. Furthermore, in order for an implicit
method to achieve the monolithic solution (assuming its convergence is given, i.e. the before mentioned
stability limit is not exceeded) the number of subiterations must be increased as time step size decreases
([15], [42]).