\chapter{Conclusions}
\label{cha:conclusions}
%\markboth{Conclusions}{}
%\addcontentsline{toc}{chapter}{Conclusions}

In this thesis, we  developed an adapter for \acrshort{mbdyn} to enable fluid-structure interaction simulations with the coupling library \acrshort{precice}.
The adapter has been validated both qualitatively and quantitatively with several test cases.

This work allows MBDyn to extend its capabilities in \acrshort{fsi} simulations, as it can be coupled with a wide set of \acrshort{cfd} solvers which are already connected to preCICE.

Among other possible simulations, this connection allows the \acrshort{fsi} simulation of slender bodies immersed in fluid flow and gives the possibility of prescribing different structural properties independently from the external geometry: shape and structural properties of the structure can be decoupled, thus allowing specific and independent tuning of parameters.

The adapter introduced here can also be used with other \acrshort{mbdyn} elements (e.g. shells), simply changing the MBDyn configuration file. Besides, it has been developed independently from MBDyn itself as it makes use of external \acrshort{api}s. So it does not impact directly the development of MBDyn.

During the development phase, a great attention has been given to the configuration process of a simulation. For this reason, most of the adapter parameters are read at runtime from a \acrshort{json} configuration file.  

Given the development and the tests carried out in this work, some directions of further analysis and development appear to be particularly interesting.
First of all it is necessary to better understand the limits of application for the whole \acrshort{fsi} setup as it is now: for example if there exists a set of parameters that allows to simulate models similar to the FSI3 benchmark, as it is known to work by simply using a different structural solver.

It will also be important to analyze the behavior and the performances of a simulation in more complex, 3D real world scenarios, where more computational resources are needed.

There can also be future developments in the software code of the adapter itself. Some of them concern the whole simulation: for example it would be useful to  make the configuration easier and less redundant, as some of the parameters are duplicated among the components.

Other extensions are relative to the possibility of applying a variable time step to the \acrshort{mbdyn} simulation: in principle, the fluid and the solid simulation can advance at different time steps, as they might have different specific needs. But in this case, at some point in the simulation, \acrshort{precice} will need to synchronize the two simulations by forcing a prescribed time step to the solvers. For our simulations we kept the time step coincident among the solvers: the fluid solver generally requires a shorter time step and the \acrshort{mbdyn} simulation is not costly. But there can be situations in which this feature might be useful. It has already been partly developed and it does not require much work to be finalized.

Another extension concerning the adapter consists in giving preCICE information about the topology of the mesh (e.g. triangles or quadrangles defining the interface). This would allow a better use of some of the mapping strategies. This information is already available in the adapter as it is used to draw output information and should be easy to send it also to preCICE.

Anther, more complex, development would consist in possibly extending MBDyn capabilities to simulate \acrfull{ssi} applications. This would allow to couple MBDyn, though preCICE, with different \acrshort{fem} solvers (or even with itself). This kind of coupling would allow to simulate multibody models, in which some relevant sub-models are simulated by means of a dynamic FEM model. This kind of development would require some modifications also in the MBDyn code.

Some of the preliminary results have been already presented to the \textit{second preCICE community hour}, but for the continuation and the future development of this work we aim at including the code, together with some usage examples, into the MBDyn and preCICE repositories and websites, to find other benchmarks, users, comparisons and use scenarios.


%This work gives insights into the basics of FSI simulations and allows to classify the implemented partitioned
%coupling among the wide variety of coupling strategies
%
%The adapter that we propose applies to a wider set of OpenFOAM solvers, is more
%configurable, and can be plugged in at runtime without requiring any modifications to the
%solver’s source code – an extra step that previously raised difficulties for the user.
%
%
%Our proposed adapter not only allows for higher compatibility, but also makes the preparation
%stage easier because it can simply be “plugged-in” at runtime. Up to now, this
%required additional changes in the solver’s source code. 
%
%Our adapter is also more flexible. The solver-specific parameters are read from the
%OpenFOAM configuration files (dictionaries) and the user does not need to specify them
%elsewhere. The boundary conditions on the interfaces are configured through the adapter’s
%configuration file (YAML) and have been extended to support more solvers. The adapter
%can now directly read any additionally required material properties from the case files,
%instead of requiring modifications to the solver to read them. In case a solver uses different
%names for the required fields, these are also configurable.
%
%We validated the proposed adapter with multiple solvers, for two scenarios. The first
%one was a flow over a heated plate, described in [19.]. In most of the cases, the result files
%using the proposed adapter were identical to the files that were produced using the previous
%adapters. In some cases, sporadic errors in the order of the results’ precision were observed.
%In the case of the buoyantBoussinesqPimpleFoam, a problem was discovered and corrected
%in the previous implementation, which was an example for the need for checkpointing in
%implicit coupling. The second scenario was a shell-and-tube heat exchanger, a complex
%
%This work makes preCICE compatible with a wide range of OpenFOAM solvers used
%widely in both academia and industry. This brings preCICE not only closer to users but
%also to developers. Being designed with extensibility in mind, our proposed adapter could
%be equipped with additional boundary conditions for mechanical fluid-structure interaction
%or other computational fields, making multi-physics simulations more accessible. In this
%direction, the adapter could be ported to older versions and other variants of OpenFOAM.
%Test cases should be developed, in order to make the continuous development of the adapter
%for multiple OpenFOAM versions easier. Additionally, tutorials would help to reach new
%users. In order to convince the OpenFOAM community, examples that highlight the advantages
%of preCICE over the monolithic OpenFOAM solvers for conjugate heat transfer
%should be presented. Developing additional modules for mechanical fluid-structure interaction
%or acoustics would help to establish preCICE and OpenFOAM as standard components
%of multi-physics simulations, leading to benefits for both communities










%An adapter for linking the CFD solver SU2 with the multiphysics coupling tool preCICE is developed in
%this work and validated both qualitatively and quantitatively with several testcases.
%This work gives insights into the basics of FSI simulations and allows to classify the implemented partitioned
%coupling among the wide variety of solver strategies. The integration of the adapter into the code
%structure of SU2 is explained, as well as its implementation itself. The coupling approach is validated
%via generic two- and three-dimensional testcases, and the well-known numerical benchmark scenario FSI3
%([39]). The adapter is capable of handling three-dimensional, real-world examples. This is exemplified
%by simulating a long, slender cylinder under turbulent flow conditions. The practical background of this
%simulation is research on brush seals for e.g. gas turbines. However, time did not suffice to obtain meaningful
%simulation results in the scope of this thesis. Thus, work on this topic will be continued. Once
%feasible single-cylinder simulation results are achieved, scenarios involving multiple cylinders in a channel
%should be simulated in order to include possible mutual interactions of these structures. Thereby, the
%real conditions in brush seals can be approximated more precisely. In its current version, the adapter
%is limited to a single wet surface in the SU2 mesh file. While it is technically possible to define the
%surface of multiple cylinders as a single wet surface, it is more accurate to extend the adapter such that
%an arbitrary number of separate FSI interfaces can be defined. Moreover, it might be useful to integrate
%a load ramping functionality in the adapter, such that at the beginning of an FSI simulation the forces
%exerted on the structure are reduced by a factor, which is consequently increased to recover the real
%force values after a certain number of time steps. This might help to stabilize simulations as the initially
%occurring, large displacements can be reduced by this method. Recently, several RBF mapping methods
%have been included in preCICE, which can be used for parallel simulations. Applying these elaborate
%interpolation procedures instead of NN mapping allows to use finer solid meshes without encountering
%oscillatory forces at the wet surface, such that simulation results are expected to significantly improve
%(compared to simulations done with NN mapping). Furthermore, the spatial oscillations at the FSI interface
%are expected to vanish. Consequently, further testing with the FSI3 benchmark scenario will be
%done in order to verify this.
%The adapter is implemented such that it perfectly integrates into SU2, allowing to reuse characteristic
%functionalities like turbulence modeling, convergence acceleration via dual time stepping and multi-grid
%functionalities. Moreover, usage of preCICE for coupled FSI simulations can be configured via the native
%SU2 configuration file. A single executable of SU2 can be used for different FSI scenarios and even
%fluid-only computations without the need for recompilation. The adapter allows for inter- and intrafield
%parallel simulations. While no ALE implementation is currently available for the incompressible solver
%in SU2, the adapter is already prepared for this feature, possibly extending the FSI capabilities to the
%incompressible regime in the future.
%Although intrinsic FSI functionalities were recently added to SU2, the coupling with preCICE offers more
%flexibility, as even commercial partner solvers can be chosen for multiphysics simulations. Moreover, the
%coupling algorithms and data mapping methods of preCICE are more elaborate. The possibility of
%running fully-parallel simulations outperforms the intrinsic FSI capabilities with regard to HPC.
%Another practically very useful outcome of this work is the description of the procedure for integrating the
%adapter into SU2 and building it with preCICE. Thus, users can easily extend SU2 to FSI via preCICE
%guided by this thesis, which addresses both SU2 and preCICE communities. The adapter will soon be
%included in the preCICE repository.