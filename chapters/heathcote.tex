
\section{Flapping Wing simulation}

An application example of \acrshort{precice}-\acrshort{mbdyn} coupling in the study of \acrshort{fsi} problems can be found in \cite{heathcote2008effect}, in which the effect of spanwise wing flexibility on thrust, lift and propulsive efficiency of a rectangular wing oscillating in pure heave is analyzed by means of water tunnel experiments.

The study shows that, for some oscillating frequencies, a degree of spanwise flexibility yields a small
increase in thrust coefficient and a small decrease in power-input requirement, resulting in higher overall efficiency.

\subsection{Experimental setup}

Before describing the \acrshort{fsi} model and the simulation, it is necessary to briefly introduce the experimental setup.

The study considers three types of rectangular wings, with profile \textit{naca 0012} and with different section properties, as shown in Figure~\ref{fig:profiles0012}. The section labeled (i) is considered \textit{inflexible}, the one labeled (ii) is considered \textit{flexible} and the last one \textit{highly flexible}.

\begin{figure}[htbp!]
	\centering
	\includegraphics[width=0.7\textwidth]{images/profiles0012}
	\caption{wing section properties (image taken from \cite{heathcote2008effect})}
	\label{fig:profiles0012}
\end{figure}

Each wing has the following dimensions: $100$\si{mm} chord and $300$\si{mm} span.

The experimental setup is shown in Figure~\ref{fig:0012exp}. The displacement of the root section is given by $s = a_{ROOT} \cos(\omega t)$, where $a_{ROOT} = 0.175c$. The flow velocity $U_0$ is in the range $1-3$\si{m.s^{-1}}. The following dimensionless parameters are considered:

\begin{itemize}
	\item $Re = \frac{\rho U_0 c}{\mu}$: Reynolds number
	\item $k_G = \frac{\pi f c}{U_0}$: Garrick reduced frequency
	\item $S_r = \frac{2f a_{MID}}{U_0}$: Strouhal number at mid-span
\end{itemize}

\begin{figure}[htbp!]
	\centering
	\includegraphics[width=0.7\textwidth]{images/naca0012_exp}
	\caption{experimental setup (image taken from \cite{heathcote2008effect})}
	\label{fig:0012exp}
\end{figure}

Experiments are carried out for the three types of wings in the following ranges: $Re=1\cdot10^4-3\cdot10^4$ and $k_G=0-7$.

The results give information concerning the average thrust coefficient $C_T = \frac{T}{\frac{1}{2}\rho U_0^2c}$ over a finite number of cycles and the mean power input coefficient $\bar{C}_P = \frac{\bar{F_y v}}{\frac{1}{2}\rho U_0^3c}$.

Besides, information concerning the ratio $\frac{a_{TIP}}{a_{ROOT}}$ and tip phase lag $\phi$ are given.

\subsection{Simulation setup}

The experimental setup described in \cite{heathcote2008effect} has been replicated in a \acrshort{fsi} simulation using \acrshort{mbdyn} as structural solver and OpenFOAM as \acrshort{cfd} solver.

\subsubsection{Fluid domain}

The fluid domain is represented by a box of size $1.5\times 0.6\times 0.5$\si{m}, as represented in Figure~\ref{fig:hc-mesh}. The boundary conditions are:

\begin{itemize}
    \item constant inlet velocity for the face at $x=-0.25$\si{m},
    \item constant pressure for the face at $x=1.25$\si{m},
    \item symmetry plane at the root of the wing ($z=0$),
    \item slip walls for the other external surfaces,
    \item no-slip wall for the wing surface.
\end{itemize}

The naca0012 wing has been drawn in \textit{Salome} and exported in OpenFOAM as \textit{.stl} file. The mesh has been built with the tool \textit{snappyHexMesh} and it is composed of 218451 cells. 



\begin{figure}[htbp!]
	\centering
	\begin{subfigure}{.75\textwidth}
		\centering
		\includegraphics[width=.99\linewidth]{images/heathcote/dom03.png}
		\caption{mesh bounding box}
		%\label{fig:undist}
	\end{subfigure}
	\newline
	
	\centering
	\begin{subfigure}{.75\textwidth}
		\centering
		\includegraphics[width=.99\linewidth]{images/heathcote/dom02.png}
		\caption{wing detail}
		%\label{fig:dist}
	\end{subfigure}
	\caption{fluid mesh}
	\label{fig:hc-mesh}
\end{figure}



\subsubsection{Interface}

The same naca0012 profile drawn in Salome has been used to generate the interface mesh for the \texttt{external structural mapping} of MBdyn. The interface mesh is composed of 1286 cells (1200 quadrangles and 86 triangles), as shown in Figure~\ref{fig:hc-interface}.

\begin{figure}[htbp!]
	\centering
	\includegraphics[width=0.75\textwidth]{images/heathcote/interface01.png}
	\caption{interface mesh}
	\label{fig:hc-interface}
\end{figure}


\subsubsection{Structural domain}

The structural model is composed of 10 MBDyn \texttt{beam} elements. Considering the \textit{flexible} (or the \textit{highly flexible}) structure, the stiffness of the wing can be considered completely given by the 1\si{mm} metal plate, as the PDMS, with a Young modulus of $360\div870$ \si{kPa} would contribute with only a small fraction of the overall stiffness.

The stiffness matrix of each beam element is given in Equation~\ref{eq:hc-stiff}, in which $w$ represents the chord and $h$ the metal thickness.

\begin{equation}
    \begin{bmatrix} Ewh &  & 0 & \ldots &  & 0 \\
                          & Gwh &  & & & &  \\
                          & & Gwh & & & \\
                          & & & G\frac{1}{3}wh^3 & & \vdots \\
                          & &  sym. & & E\frac{1}{12}wh^3 &  \\
                          & & & & & E\frac{1}{12}hw^3
    \end{bmatrix} 
    \label{eq:hc-stiff}
\end{equation}

At each of the 21 nodes of the structure there is a \texttt{body} element attached, carrying the mass of the corresponding chunk of beam (both metal and PDMS).

The wing motion is achieved by means of a \texttt{total pin joint} element attached to the root of the wing, which moves the root \texttt{node} with a $\sin(\omega t)$ law along the $y$ direction. The motion starts with a ramp of the  kind $\frac{1}{2}\left(1-\cos\left(\frac{t}{\tau}\right)\right)$ in order to ease convergence.

\subsubsection{Coupling}

