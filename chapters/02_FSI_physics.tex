\chapter{Physical aspects of Fluid-Structure Interaction problems}
\label{cha:physics}


TODO: Intro

\section{Description of motion}
\label{sec:desc-motion}
A fluid in motion is considered and interaction with a solid occurs via deformations of the latter due to
viscous and/or inviscid forces exerted on the structure by the fluid. The resulting change in shape of
the structural material also implicates geometrical changes of the fluid domain. This yields different flow
behavior in reverse. Thus, it is necessary to represent kinematic and dynamic processes formally. Therefore,
some thoughts concerning the motion of the before mentioned continuum particles must be made.
In classical continuum mechanics there are two different perspectives ([26]): The Eulerian description,
which is discussed in Section 2.2.1, and the Lagrangian point of view, which I explain in Section 2.2.2.
Those two perspectives can be combined to the arbitrary Lagrangean-Eulerian (ALE) method, described
in Section 2.2.3.

\subsection{Eulerian perspective}
\label{subsec:euler}

In a Eulerian perspective the change of quantities of interest (e.g. density, velocity, pressure) is observed
at spatially fixed locations. In other words, a Eulerian observer does not vary the point of focus during
different time steps. This is depicted intuitively in Figure 2.1. Located at a certain point in Euclidean
space, the observer always focuses on the same location, no matter where particles may move. Thus, in
Eulerian description, quantities can be expressed as functions of a fixed location as well as time. This
may be denoted by:

\begin{equation}
	\Theta = \tilde{\Theta}(x,y,z,t)
\end{equation}

where $\Theta$ is a quantity of interest and $\tilde{\Theta}$ denotes it in a Eulerian point of view. $(x, y, z)$ represent a fixed position in Euclidean space and t refers to time. Clearly, different particles can occupy the spatial location, which the observer focuses on, at different instances of time. Therefore, in general no direct information regarding the change of quantities of a single particle is available when motion is described
in a Eulerian perspective ([26]).
Eventually, a description of motion is needed not only for single particles and points in space, but rather
computational domains and meshes being central aspects of FSI problems. A computational mesh can be
interpreted as a number of observers distributed across the domain of interest and connected so as to form
a grid with nodes. If particles of the underlying domain move, a purely Eulerian mesh does not change
the positions of its nodes throughout the whole mesh at different instances of time. This is graphically
shown in Figure 2.2. Since this behavior of the mesh is independent of large-scale movements of particles,
it is the typical choice for CFD problems, where in general fluid particles move throughout the whole
computational domain. However, this approach also has its drawbacks as the level of refinement of the
mesh is crucial to the accuracy of computations because it defines to what extent small-scale changes can
be observed. If a mesh is of a much coarser scale than the motion occurring in the underlying domain,
the motion cannot be resolved ([10], [26]).

\subsection{Lagrangian perspective}
\label{subsec:lagrange}

A Lagrangian description implies that the observer focuses on a specific particle and follows it, regardless of the speed and distance it may travel. Therefore, provided that the particle moves, changes of the quantities of interest are observed at different spatial locations. The Lagrangian observer tracks a particle and moves with it, as illustrated in Figure 2.3.
The motion of the particle as well as all other quantities of interest, can therefore be described by
reference coordinates (or material coordinates) in Euclidean space, $(X, Y, Z)$, uniquely identifying the
observed particle at a reference configuration. Often $t = 0$ is chosen as reference but in general any time
instance can be used. Once the particle to be observed is specified, the Lagrangian observer only registers changes concerning this one particle as time passes. Thus, quantities of interest can be described as
\begin{equation}
\Theta = \hat{\Theta}(X, Y,Z, t)
\end{equation}
Again, computational domains and meshes are considered: At a reference instance of time, usually at the
beginning of a simulation, mesh nodes are attached to the underlying material particles. As time passes
and particles move, the mesh nodes move with them causing the mesh to deform (except for cases in
which all particles move smoothly with equal speed and distance). Figure 2.4 depicts such a situation.
As it can be seen, the mesh nodes always coincide with their respective particles. A drawback of this
Lagrangian technique is that large-scale and irregular motions lead to distortions of the computational
mesh, which yields smaller accuracy in simulations as a consequence of the strictly enforced tracking.
However, from this point of view, small-scale motions, which often occur in solids, can easily be observed
without the need of using extremely fine meshes, which would be necessary in case a Eulerian perspective
was used. This results in reduced computational effort. Therefore, in general, the Lagrangian description
is the method of choice for CSM problems ([10], [26]).
Eulerian and Lagrangian descriptions are related. A mapping between them can be derived if the motion
is known:
\begin{equation}
x_i = X_i + u_i(X_i, t) \quad \forall i = 1, 2, 3
\end{equation}
Equation 2.4 can be explained as follows: The Eulerian, spatial position x of a particle at time t is the
position of this particle at its reference configuration X plus the displacements u that it traveled since
the point of time of the reference state ([26]).

\subsection{ALE method}
\label{subsec:ALE}

Finally, I explain the ALE approach, a combination of the Eulerian and Lagrangian perspective widely
used for FSI problems. As the name implies, an ALE observer can arbitrarily decide whether to move
the point of focus or not. Furthermore, the observer is in no way restricted to the movement of particles.
Figure 2.5 depicts such a situation. The observer moves independently of the particle motion.
By analogy with the Eulerian and Lagrangian meshes before, an ALE mesh is considered as it can be seen
from a Eulerian perspective in Figure 2.6. Mesh nodes can move almost arbitrarily regarding the motion
of the underlying particles. The only restriction is, that node movements should not distort the mesh too
much as this leads to inaccuracy. It is reasonable to allow the nodes to follow moving particles up to a
certain extent, which is defined by mesh quality criteria. Since this approach does not allow to directly link mesh motion and material particle motion, a new unknown is introduced to such a problem, namely
the relative movement between the ALE mesh and the material domain. This approach is especially
interesting for FSI problems because it is an alternative description to the Eulerian frame for the fluid
domain. As it is further explained in Section 2.3.3, fluid and solid material have to follow the moving
interface between them for physical reasons. Since the solid domain is usually described in a Lagrangian
view, there is no problem with keeping the solid mesh attached to the FSI interface. However, if a purely
Eulerian approach was used for the fluid domain, movements of the interface would lead to gaps between
the wet surface and the fluid mesh. Therefore, in ALE methods the fluid mesh nodes at the interface
always move with it. This can be interpreted as Lagrangian fashion of the approach, as fluid mesh nodes
follow the fluid particles sticking to the interface, while the rest of the fluid mesh is allowed to move
in such way that mesh distortions are kept minimal in order to preserve computational accuracy. Since
preserving mesh regularity refers more to a Eulerian approach, the choice of the name ALE becomes
apparent ([32], [10]).


\section{Domains and interface}

As the name fluid-structure interaction implies, this type of problems is determined by the fluid and solid
domain, covered in Sections 2.3.1 and 2.3.2, respectively. Furthermore, their interaction is of importance,
which underlines the necessity of suitable coupling conditions at the domain common interface. The
interface is also referred to as wet surface. Its formal definition is stated in Section 2.3.3

\subsection{Fluid domain}

In the following, all of my considerations are limited to viscous Newtonian flows in the compressible
regime as this kind of model is the only relevant one for this thesis. Nevertheless, I want to point out that
throughout the FSI community also incompressible and inviscid flow regimes are commonly considered,
depending on the type of physical problem.
The before mentioned kind of flow is described by the Navier-Stokes equations (NSE), which I consider
in the general three-dimensional case in a Eulerian description. They consist of the continuity equation
(conservation of mass, Equation 2.5a), the momentum equation (conservation of momentum, Equation
2.5b) and the energy equation (conservation of energy, Equation 2.5c). The equations are shown in index
notation. Repeated indices imply Einstein’s summation convention. For a detailed explanation of this
convention, I refer to [26]. The NSE are usually derived by applying Newton’s Law to a fluid control
volume and an elaborate derivation can be found in [14]. The equations are taken from [26] and [14].

\begin{eqnarray}
	\rho_t + \left(\rho u_j\right)_j &=&  0 \\
	\left(\rho u_i\right)_i + \left(\rho u_i u_j +p\delta_{ij} -\tau_{ij}\right)_j &=& 0 \quad \forall i,j = 1,2,3 \\
	\left(\rho e_0\right)_t + \left(\rho e_0 u_j +u_jp + q_j -u_i \tau_{ij}\right)_j &=& 0
\end{eqnarray}


$\rho$ denotes density, t time, \textbf{u} flow velocities in all dimensions and p pressure. $\delta_{ij}$
is the Kronecker delta, $\tau$ the viscous stress tensor, $e_0$ total energy (per unit mass) and \textbf{q} heat flux (via conduction). For a Newtonian fluid the viscous stress tensor is given by

\begin{equation}
	\tau_{ij} = -\frac{2}{3}\mu u_{k/k} \delta_{ij}+2\mu S_{ij} \quad \forall i,j = 1,2,3
\end{equation}


With $\mu$ being the dynamic viscosity and \textbf{S} the rate of deformation tensor (symmetric part of the velocity gradient $\nabla u$):

\begin{equation}
	S_{i,j} = \frac{1}{2}u_{i/j}+u_{j/i} \quad \forall i,j = 1,2,3
\end{equation}

In order to form a closed set of these partial differential equations (PDE), it is necessary to choose a
conductive heat flux model (usually Fourier’s Law), specify the caloric and thermodynamic equations of
state and finally, choose appropriate initial and boundary conditions for the problem ([17], [14], [26]).

Simplification can be done to obtain easier models such as: adiabatic, inviscid, incompressible...


\subsection{Solid domain}

As described in Section 2.2.2, in solid mechanics usually a Lagrangian point of view is used (as is here),
because particles do not travel as far as they do in fluid dynamical problems. Also, the structural model
explained in this section is limited to the Saint Venant-Kirchhoff model, which is very common since
it is capable of handling large deformations often occurring in FSI problems ([18]). The model assumes
that the solid material is homogenous, meaning that mechanical properties of a particle of the body do
not depend on the location of the particle. In other words, these properties are the same throughout the
whole solid domain. Moreover, isotropy is assumed, such that the direction in which a stress is applied
to the solid does not matter, as the mechanical properties of the body are the same in all directions ([26],
[45]).

The following explanation is a short version, since this thesis does not focus on the solid mechanical
aspects of FSI problems. It is inspired by and partly taken from [18] and [26], where derivations are given
to a more detailed level.
By analogy with the NSE (see Equations 2.5), the description of the solid arises from considering a control
volume and applying Newton’s Law to it. A typical equation of motion in the form Mass  Acceleration
= Forces can be derived (again, the general three-dimensional case is considered):

\begin{equation}
	\rho u_{tt} = S_{ij/j} + \rho f_i \quad \forall i,j = 1,2,3
\end{equation}

Here,  corresponds to the structure density, the second derivative of the displacements u with respect
to time t to the acceleration of a material particle and S to the second Piola-Kirchhoff stress tensor. X
denotes the material coordinates as mentioned before. In this case, also the volume force f is considered,
because gravity can often not be ignored for solid materials. Again, a constitutive law must be taken into
account, defining the relationship between stress and strain:

\begin{equation}
	S_{ij} = \lambda E_{kk} \delta_{ij} + 2\mu E_{ij} \quad \forall i,j,k = 1,2,3
\end{equation}

with

\begin{equation}
	E_{ij} = \frac{1}{2}\left(\frac{\partial u_i}{\partial X_j}   + {\partial u_j}{\partial X_i} \right) + \frac{1}{2} \frac{\partial u_k}{\partial X_i}\frac{\partial u_k}{\partial X_j} \forall i,j,k = 1,2,3
\end{equation}

Note that E is the Lagrangian (finite) strain tensor. The latter summand in Equation 2.10 is nonlinear
and can be neglected for small deformations, leading to the Lagrangian infinitesimal strain tensor.
However, since we deal with possibly large deformations, this relationship remains non-linear. ij again
refers to the Kronecker delta.  and  are material parameters named Lamé constants. They relate
directly to Young’s modulus E and Poisson ratio , which are of more practical use1. Their relationship
is given as follows:

\begin{eqnarray}
	E &=& \frac{\mu(3\lambda+2\mu)}{\lambda + \mu} \\
	\nu &=& \frac{\lambda}{2(\lambda + \mu)}
\end{eqnarray}


Either (E; $\nu$) or ($\lambda$; $\mu$ ) are enough to fully characterize this material under specific assumptions:

\begin{itemize}
	\item The solid is linearly elastic and isotropic. 
	\item The strain tensor E is symmetric,
	\item as well as the stress tensor S. 
\end{itemize}

 
 Furthermore, a scalar, positive definite strain energy density function (not shown in this shortened
explanation) relating stress and strain tensor via a potential formulation exists.
For more sophisticated explanations the interested reader may refer to [26].

TODO beam model

\subsection{Interface and interaction}

Since FSI problems are centered on the interaction of the fluid and solid domain, their common interface
is of vital importance. A schematic picture of a sample situation at the wet surface is shown in Figure
2.7. Note that all quantities related to the solid and fluid domain, as well as the interface are subscripted
with S, F and FS, respectively. Also, in order to avoid simultaneous usage of sub- and superscripts, I
switch from index to direct notation in this section. In order to couple both domains via the interface
in a physically correct way, some conditions must be met. These conditions are commonly used for FSI
problems. However, in this specific case they are taken from [18] and [20].
First of all, fluid and solid domain should neither overlap, nor separate from each other at the interface
as there can be no space occupied by fluid and solid particles at the same time and "empty" space is non-physical. Furthermore, for a viscous fluid the flow velocity at the domain boundary has to be equal
to the boundary velocity itself, which is called no-slip condition. Together, this results in the kinematical
requirement that the displacements of fluid and solid domain, as well as their respective velocities have
to be equal at the wet surface (denoted by $\Gamma_{FS}$):

\begin{eqnarray}
 \vec{x}_F &=& \vec{u}_S \\
 \vec{v}_F &=& \frac{\partial \vec{u}_S}{\partial t}
\end{eqnarray}

For inviscid fluids only velocity components normal to the wet surface have to be equal to the structural
velocity as the fluid may slip freely in tangential direction at any boundary.
It is not sufficient to consider only kinematic constraints at the interface. In addition, an equilibrium
of forces at the wet surface is needed such that it is not torn apart by resultant forces. Force vectors
originate from the stresses at the interface and the outward normal vectors of fluid and solid domain,
respectively. They have to be equal and opposite leading to the dynamic coupling condition:

\begin{equation}
\sigma_F \cdot \hat{n}_F = -\sigma_S \cdot \hat{n}_F
\end{equation}

$\sigma $ 2 R33 denotes the stress tensor and n 2 R3 the outward normal vector of the fluid and solid domain.
Note that here viscous as well as inviscid stresses are included.

\section{Classification of FSI problems}

you have been interested in the effect of boundary conditions on the flow. For instance, here is the effect of a cylinder that deviates a uniform flow. In fluid mechanics, we consider solids as boundary conditions only,
and not in terms of what they are made of.
In solid mechanics, we usually consider fluids just as the cause of a loading at the boundary, a force type boundary condition. These two approaches are very useful, and I extensively used in engineering. For instance, in civil engineering you may find engineers that compute wind loads on a bridge. And then send them to other engineers that will check that this is acceptable in terms of the solid mechanics of the bridge. This is often quite sufficient.
We mean situations where you cannot solve these two problems independently.
Here schematically, the cylinder is deformed by the flow, which itself is modified by the deformation of the cylinder. This is coupled fluid and solid mechanics. Now, when you think of it, this question of coupling of models is actually quite fundamental.

First, find a way to classify all these couplings. Why? Because the variety seems so large that it does not seem feasible to find a model that is applicable to all of them. The second objective, once we have classified them is to try and build relevant models for these classes.

So when you want to go from considering dimensional quantities to dimensionless ones, what can you do?
There is a rather general theorem called the Pi Theorem or the Vaschy-Buckingham Theorem, which tells you how many dimensionless quantities you need to look for.
This theorem states that the number of dimensionless quantities, P, is equal to that of the dimensional ones, N, minus R
What is R? It is the rank of the matrix of dimension exponents. This matrix is formed by the columns of the dimension exponents of all variables, as you can see here. Remember that the rank of a matrix is a number of independent lines or columns that you can find. Let us give an example


\subsection{Dimensional analysis}

CFR w1-2

we need to classify all these cases of mechanical coupling between fluids and solids. 
The tool we shall use is called Dimensional Analysis.
It is dimensionless in the sense that we do not need any scale of unit to express it. It is just a number.
Here, let us just take as a principle that a physical law should only relate dimensionless quantities.
There is a rather general
theorem called the Pi Theorem or the Vaschy-Buckingham Theorem, which tells you how many dimensionless
quantities you need to look for. This theorem states that the number
of dimensionless quantities, P, is equal to that of
the dimensional ones, N, minus R. What is R? It is the rank of the matrix
of dimension exponents. This matrix is formed by the columns of
the dimension exponents of all variables, as you can see here.
Let us consider, schematically,
the fluid here in blue and a solid in red. To make things simpler, we assume that
they stand in separate domain of space and that there is no mass
transfer between them. As for the case of drag on a sphere,
we now need to specify what quantities we want to use to define
our problem and what we are looking for. First, the fluid on the left. Let us say that we're looking for
the local velocity U in relation to the coordinate of the point
we consider X and the time T. This is also going to depend on
the viscosity of the fluid, mu, The density of the fluid rho and
the gravity, G. Also the result is going to be different
if I change the size of the domain. So I say the result depends on the size L. Of course, this velocity is also going to
depend on some boundary condition. For instance, an upstream flow
velocity that I call U naught. This is the list of quantities I'm
considering in a given problem in the fluid. Second, the solid on the right. We might want to know the displacement,
csi, at a position X, at a time T. It may depend on E,
the stiffness of the solid, (I shall come back to this later) on it's density, rhoS and on gravity G. Again, it will also depend on the size. And there is somewhere the magnitude of this displacement that is set,
say cdi naught. As you can see here,
I'm quite general in stating what the problem is. Still by stating that this is
the list of quantities that I want to relate by my physical law,
I'm not that general. For instance,
I've excluded the temperature. But we have to choose what is the kind of
problems that we want to consider. And this is already quite general.

\subsection{Dimensional analisys in fluid domain}

CFR w1-3

We are going to start by
something very simple. Doing dimensional analysis separately
in the fluid and in the solid. Imagine now that what
happens in one domain is totally independent of
what happens in the other. For instance, in the fluid. This is what you have
done in fluid mechanics when you have ignored all possible
influence of what happened inside a solid that bounds the fluid. So, we assume that there exist
a physical low that relates the fluid velocity with all the other
parameters, namely X, T and so on. This means that the flow is not going to
depend on the deformation of the solid, because the stiffness E  for instance,
is not included in there. This is pure fluid mechanics. Let us do the dimensional
analysis of this. Here is a law F between
the dimensional variables. There are eight. To use pi theorem, I need to build
the matrix of dimension exponents. Here it is. X is the coordinate, so it is a length. T is a time. U is a  length per time and so on. As soon as you can put some
units on these quantities, you can write the dimension exponents. Now, what is a rank of this matrix? We can find three independent vectors. For instance, here. And certainly no more than three
because the dimension is three. So the rank is all equal to three. I can conclude that we
should be looking for 8 minus 3 equals 5 dimensionless parameters. So, let us write the law
we are looking for in the form of one depending on
only five dimensionless parameters. What are these dimensionless parameters? We know that we should find
five independent ones. I can easily start by defining a dimensional
velocity by diving U by U naught. Both are velocities. So the ratio is dimensionless. Second, X divided by L. Third, something I shall
explain in a moment. Then, of course, the Reynolds
number that combines these four. What else? I haven't used the gravity G so far. So let us use it in
a dimensionless number. Here is what is usually called the Froude
number combining U naught, G and L. These five members are dimensionless and
they are independent. You cannot get one by
a combination of the others. Let us go back to the ratio
U naught T over L. As all dimensionless quantity,
this one can be understood as the ratio of two dimensional quantities,
two lengths, two times. I can write this one as T over T fluid
where T Fluid equals L over U naught. What is L over U naught? It is just the time taken by
a particle of velocity U naught to travel across the distance L. So T fluid is a time scale associated
with convection in the fluid. A very important quantity
that we shall use later. At this stage, we have just written
down the fact that the dimensionless velocity in  the fluid is dependent
on a dimensionless coordinate, a dimensionless  time,
the Reynolds number, the Froude number.

\subsection{Dimensional analysis in solid domain}

Let us now do the same for
the solid alone. Now, we look for a relation between
all quantities on the solid side. F of X,T, csi, E, L, G, tho s chi naught, equals zero. I have singled out the displacement, which is unknown. 
Let us use again the pi theorem. Here is the matrix of
the dimension exponents. We have here, too,  8 quantities, a rank of 3, and so 5 dimensionless parameters to find. What are they? Here is a choice. The dimensionless  displacement where
I've divided csi by the length L. The dimensionless coordinate or
dimensionless time , I will discuss just after, and
two other dimensionless parameters. The first one is the ratio between
the displacement data csi naught  and the length scale of the system. We shall call it the displacement number. When large, the displacements
are large with regard to the size. This is what we call
usually large displacements. The second one combines gravity,
density, length and stiffness and I shall call it
the elastogravity number. When it is large, it means that the deformations induced 
by gravity in the solid are large. For instance, in a jelly cake,
the shape is really effected by gravity. Let us go back now to the dimensional
time that I introduced. I can write this as T over Tsolid,
where Tsolid is L over a velocity C, and this velocity us
square root of E over rho s. What is it? It is actually the scale of elastic
wave velocities inside the solid. So T solid is the time that an elastic
wave takes to go across the solid.


\subsection{Dimensional analysis of coupled problems}

cfr W1-4

We are now ready to undertake
the dimensional analysis of a fully coupled fluid and solid interaction problem. We have done the case of the fluid alone and
the case of the solid alone We're going to use exactly the same
method but considering the fluid and the solid, simultaneously. We are back to our full list of
parameters that define the problem. Let us discuss a bit what
these quantities are. Some of them are only defined on
the fluid side, or on the solid side. This is, for instance, the case of
the viscosity mu in the fluid or the stiffness E in the solid. Others are common to both domains such as
the gravity g or the scale of lengths L. What about our variables of interest, those that we want to
relate to the parameters? I mean, the velocity U or
the displacement csi. One of them is defined in the fluid and
the other in the solid. But now, we are going to consider
that they are related  to all the parameters of the problem without separation.

What are these dimensionless quantities
in such of problem that mixes fluid and solid ? Let us try to give a set of eight
independent dimensionless quantities out of the 11 dimensional ones. I'll start with the one I know. U over U naught, x over L, U naught t over L, the Reynolds number and the Froude  number. That makes five. Now, I  can also use the ones I
know from the solid side, combining the three quantities in a solid,
and that gives us the displacement number,
csi naught  over L, and the elastogravity   number, G. That makes 5 + 2 equals 7. But from the pi theorem I know I should
use eight dimensionless quantities. 

It necessarily mixes things from
the fluid and the solid side otherwise I would have found
it before when doing the uncoupled case. So what is it? What can we imagine as the dimensionless
quantity combining fluid and solid dimensional once.

\subsubsection{Mass number}

 The simplest one is the ratio
of the two densities. Let us call it the Mass Number, M. This seems a very good choice because it
simply tells you that it is different for a solid to interact with air or
with water. In the hard-disk drive example, M is the  order of 1,  air,
over 10 to the 4, metal, and so M is of the order of 10 to the minus 4. Conversely, for the dolphin skin, both media have about the same density,
and M is the order of 1. 

\subsubsection{Reduced velocity}

Here is another possible choice,
the reduced velocity. It is the ratio between our free velocity,
U naught, and the velocity of elastic waves in a solid, c. This also seems a good idea
because it contains information on the way the two dynamics are related. It would be quite different between
two examples I considered before. The inflatable dam and the dolphin's skin. [MUSIC] As possible new dimensionless parameters I
have proposed the ratio of two densities, that was the mass number and the ratio of
two velocities, that was the reduced  velocity.

\subsubsection{Cauchy number}

 I can also imagine something
combining stresses or stiffnes. This here is the Cauchy number. What does it mean? It is the ratio between the fluid loading
quantified by the dynamic pressure over a  unit square and
the stiffness of the solid E. 
The higher it is, the more the solid
is elastically deformed by the flow.


 These three are actually the most
important ones, and are used a lot. Which one should you choose for
your problem? Well as I said before, there is no
good choice of dimensionless numbers. But there are efficient choices, that would be more helpful
in solving a given problem. 



Figure~\ref{fig:first_figure}.
\begin{figure}[htbp!]
\centering
\includegraphics[height=0.24\textheight]{images/logo_poli}
\caption{First Figure}
\label{fig:first_figure}
\end{figure}
\\
